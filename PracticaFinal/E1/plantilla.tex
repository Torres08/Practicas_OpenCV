\documentclass{article}

% Add necessary packages
\usepackage{natbib, graphicx, fancyhdr} % For managing references
% natbib for bibtext

% Add necessary commands
\bibliographystyle{plainnat} % Specify the bibliography style

% Define margins
\setlength{\topmargin}{-1.0cm}
\setlength{\oddsidemargin}{0.1cm}
\setlength{\textwidth}{16.5cm}
\setlength{\textheight}{23.0cm}

\graphicspath{{images/}} %configuring the graphicx package

% Define header and footer
\pagestyle{fancy}
\fancyhf{}
\lhead{{\includegraphics[height=.65cm]{etsiit-1.png}}}
\rhead{\textbf{\textit{First Formal Progress Review}} }
\cfoot{\textbf{\textit{\thepage}}}
% \lfoot{\textbf{\textit{Page \thepage/\pageref*{LastPage}}}}
% \rfoot{\textbf{\textit{Alejandro Romero Prieto}}}
% \renewcommand{\footrulewidth}{0.7pt}
\renewcommand{\headrulewidth}{0.7pt}
\setlength{\headheight}{23pt}

% This is to define a style with no footer for the table of contents
\fancypagestyle{nofooter}{%
	\fancyfoot{}%
}




% Add necessary packages

% Add necessary commands
\bibliographystyle{plainnat} % Specify the bibliography style


% Define margins
\setlength{\topmargin}{-1.0cm}
\setlength{\oddsidemargin}{0.1cm}
\setlength{\textwidth}{16.5cm}
\setlength{\textheight}{23.0cm}

\graphicspath{{images/}} %configuring the graphicx package

% Define header and footer
\pagestyle{fancy}
\fancyhf{}
\lhead{{\includegraphics[height=.65cm]{etsiit-1.png}}}
\rhead{\textbf{\textit{UniQuest AR}} }
\cfoot{\textbf{\textit{\thepage}}}
% \lfoot{\textbf{\textit{Page \thepage/\pageref*{LastPage}}}}
% \rfoot{\textbf{\textit{Alejandro Romero Prieto}}}
% \renewcommand{\footrulewidth}{0.7pt}
\renewcommand{\headrulewidth}{0.7pt}
\setlength{\headheight}{23pt}

% This is to define a style with no footer for the table of contents
\fancypagestyle{nofooter}{%
  \fancyfoot{}%
}



% Title Page %%%%%%%%%%%%%%%%%%%%%%%%%%%%%%%%%%%%%%%%%%%%%%%%%%%%%%%%%%%%%%%%%%%%%%%%%%%%%%%% 

\begin{document}
\begin{center}
  \vspace*{0.5\baselineskip} % Reduced space
  \includegraphics[width=0.8\textwidth]{ugr-2.png}\\
  \vfill
  %\vspace*{8\baselineskip} % Reduced space
  {\Huge \textbf{Entrega 1 CUIA}}\\ % Enlarged font
  \vspace*{3\baselineskip}
  {\LARGE \textbf{UniQuest AR}}\\
  \begin{large}
    \vspace*{2\baselineskip}
    Fecha: 07/03/2024\\
    \vspace*{1\baselineskip}
    \emph{por} \\[1ex]
    {\Large Torres Ramos, Juan Luis \\}
	\vspace*{0.5\baselineskip}
    {{ g20596044@correo.ugr.es}}\\[1cm]
    %\vspace*{4\baselineskip}
    \vfill
	{Universidad de Granada}\\
    {\large ETSIIT}\par
	\vspace*{3\baselineskip}
  \end{large}
  \thispagestyle{empty} 
\end{center}
\pagebreak





% Contents %%%%%%%%%%%%%%%%%%%%%%%%%%%%%%%%%%%%%%%%%%%%%%%%%%%%%%%%%%%%%%%%%%%%%%%%%%%%%%%%%%%%%%%%

% \lhead{\emph{Contents}} % Set the left side page header to "Contents"
%\tableofcontents
%	\thispagestyle{nofooter}
%	%\include{abstract/abstract}
%	\cleardoublepage
%	\typeout{}

%\pagebreak

% Student %%%%%%%%%%%%%%%%%%%%%%%%%%%%%%%%%%%%%%%%%%%%%%%%%%%%%%%%%%%%%%%%%%%%%%%%%%%%%%%%%%%%%%%%%

\setcounter{page}{1}

\section{Idea Principal}
\label{sec:Idea Principal}
Descubre y conecta con tu facultad gracias a UniQuest. Haz que tu vida universitaria sea mas interactiva y sencilla mediante la implementación de AR, reconocimiento de voz y reconocimiento de imágenes. Además agrega un toque social que enriquecerá tu experiencia universitaria haciéndola mas divertida. 
\vspace*{1\baselineskip}

\raggedright
\textbf{Objetivos:}
\begin{enumerate}
	\item \textbf{Interacción en las Aulas con Marcadores:}
	  \begin{itemize}
		\item Visualizar asignaturas en 3D según el horario.
		\item Marcar la asistencia mediante la lectura de un marcador y acumular puntos por cada asistencia registrada, interactuar con AR..
	  \end{itemize}
  
	\item \textbf{Guía para Navegar entre Clases:}
	  \begin{itemize}
		\item Utilizar flechas en el suelo como guía para desplazarse de una clase a otra.
	  \end{itemize}
  
	\item \textbf{Aspecto Social:}
	  \begin{itemize}
		\item Implementar una mascota personalizable inspirada en los Mii de nintendo y avatar de reddit.
		\item Permitir que amigos escaneen tu avatar para ganar puntos sociales.
		\item Dejar mensajes de voz para tus amigos.
		\item El avatar estara enlazado con tu cuenta de la universidad.
		\item Acceder a un calendario de eventos sociales de la universidad.
	  \end{itemize}
  
	\item \textbf{Biblioteca:}
	  \begin{itemize}
		\item Escanear libros para verificar su disponibilidad.
		\item Reservar espacios de trabajo mediante la lectura de un marcador.
	  \end{itemize}
  \end{enumerate}

\vspace*{1\baselineskip}

\textbf{Tecnologías:}
	\begin{itemize}
		\item Realidad Aumentada (AR)
		\item Reconocimiento de voz
		\item Reconocimiento de imágenes
		\item geolocalización
		\item gamificación
	  \end{itemize}

\vspace*{1\baselineskip}

\textbf{Hardware y Software:}
	\begin{itemize}
	\item La aplicacion se centrara en el uso de dispositivos moviles
	\item Nos centraremos en el uso de Unity Vuforia/ARCore y java 
  \end{itemize}





  
  
  
\pagebreak



\clearpage
\end{document}